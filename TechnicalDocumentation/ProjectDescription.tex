\chapter {Definici\'on del proyecto}
	\section{Planteamiento del problema}
		\paragraph{}

	\section{Propuesta de solución}
		\paragraph{}
		
  \section {Objetivo general}
    \paragraph {Diseñar e implementar una API que brinde funciones de abstracción y operación de datos, clasificación y análisis de los mismos a través de las características de un conjunto de datos para obtener, por medio de evaluaciones y predicciones, conjuntos de objetos que representen una recomendación.}
  
  \section{Justificación}
    \paragraph{El uso de sistemas de recomendación se ha extendido popularmente en los últimos años, teniendo ejemplos en casi cualquier parte de la web, sobretodo su uso en sistemas de comercio electrónico, siendo menos comunes las aplicaciones en otras áreas. Sin embargo, el desarrollo de aplicaciones o sistemas que contengan dentro de sus características sistemas de tipo recomendación requiere un conocimiento especializado en este rubro por parte de los desarrolladores para construir el sistema desde cero, siendo al final componentes comerciales los que terminan formando parte del sistema completo, perdiendo así flexibilidad en la implementación y adecuándose el desarrollador a los requerimientos de los componentes comerciales para recomendaciones. Esto implica que al momento de desarrollar un sistema que resuelva una necesidad a través de un sistema de recomendaciones, el programador debe adquirir los conocimientos necesarios para desarrollar el sistema de recomendaciones al mismo tiempo que intenta resolver el problema de dominio al que se está enfrentando. La API a desarrollar permitirá al programador obtener las herramientas necesarias para la implementación de un sistema de recomendación utilizando conocimientos de la estadística para implementar sistemas de recomendación basados en contenido, colaborativos o híbridos dependiendo del contexto en que desee emplearlo a través de un modelo de datos adecuado que sea capaz de realizar dichos procedimientos y así reducir la curva de aprendizaje obligatoria durante la incursión en un dominio de aplicación no conocido para que este se centre en el problema que desee resolver y no en las herramientas necesarias para tal fin. Dicha API brindará funcionalidades propias de los sistemas de recomendación mediante llamadas a bibliotecas que ofrecerán el acceso a servicios desde los procesos y representará un método para conseguir abstracción en la programación. El desarrollo de una API que permita proveer la infraestructura mínima necesaria para crear un sistema de recomendación funge como proyecto integrador de los conocimientos adquiridos durante el estudio de la carrera de Ingeniería en Sistemas Computacionales, para el desarrollo del citado proyecto se requiere hacer uso de los saberes adquiridos por los participantes durante su trayectoria escolar tales como los conocimientos en materia de inteligencia artificial, ingeniería de software, matemáticas discretas, matemáticas avanzadas, reconocimiento de patrones, programación orientada a objetos, entre otras. Al final el uso conjunto de los conocimientos mencionados terminará en dicha API cuyos beneficios pueden ser vistos de manera inmediata al término de su desarrollo con su verificación y validación en un caso de estudio particular.}
