\newpage
\chapter{Anexos}
% Pruebas unitarias
% Historias de usuario
\section{Características propuestas para el caso de estudio}
  \begin{table}[h]
    \begin{center}
      \begin{tabular}{ | c | c | c | c | c | c |}
        \toprule
        Id & Tipo & Categoría & Id & Tipo & Categoría\\
        \midrule
        1 & Tipo de comida & Panes y masas  & 48 & Ocasión   & Merienda \\
        \midrule
        2 & Tipo de comida & Pastas  & 49 & Ocasión   & Cena \\
        \midrule
        3 & Tipo de comida & Bizcochos y galletas  & 50 & Región   & Italiana \\
        \midrule
        4 & Tipo de comida & Carnes  & 51 & Región   & Mediterránea \\
        \midrule
        5 & Tipo de comida & Aves  & 52 & Región   & Asiática \\
        \midrule
        6 & Tipo de comida & Pescados y mariscos  & 53 & Región   & Mexicana \\
        \midrule
        7 & Tipo de comida & Ensaladas  & 54 & Región   & Americana \\
        \midrule
        8 & Tipo de comida & Contenido alcohólico  & 55 & Región   & Hindú \\
        \midrule
        9 & Tipo de comida & Salsas y guarniciones  & 56 & Región   & Francesa \\
        \midrule
        10 & Tipo de comida & Sopas y cremas  & 57 & Región   & Tailandesa \\
        \midrule
        11 & Tipo de comida & Arroces  & 58 & Región   & Cantonesa \\
        \midrule
        12 & Tipo de comida & Legumbres y guisos  & 59 & Región   & Japonesa \\
        \midrule
        13 & Tipo de comida & Tartas y dulces  & 60 & Región   & China \\
        \midrule
        14 & Tipo de comida & Helados y sorbetes  & 61 & Región   & Medio oriente \\
        \midrule
        15 & Tipo de comida & Frutas  & 62 & Región   & Alemana \\
        \midrule
        16 & Tipo de comida & Verduras  & 63 & Región   & Argentina \\
        \midrule
        17 & Tipo de comida & Huevos  & 64 & Región   & Brasileña \\
        \midrule
        18 & Tipo de comida & Lácteos  & 65 & Región   & Colombiana \\
        \midrule
        19 & Tipo de comida & Frutos secos  & 66 & Región   & Coreana \\
        \midrule
        20 & Tipo de comida & Encurtidos y conservas  & 67 & Región   & Cubana \\
        \bottomrule
      \end{tabular}
    \end{center}
  \end{table}
  \begin{table}
    \begin{center}
      \begin{tabular}{ | c | c | c | c | c | c |}
        \toprule
        Id & Tipo & Categoría & Id & Tipo & Categoría\\
        \midrule
        21 & Tipo de comida & Postre  & 68 & Región   & Española \\
        \midrule
        22 & Tipo de comida & Bebida  & 69 & Región   & Finlandesa \\
        \midrule
        23 & Tipo de comida & Primeros platos  & 70 & Región   & Griega \\
        \midrule
        24 & Tipo de comida & Segundos platos  & 71 & Región   & Holandesa \\
        \midrule
        25 & Tipo de comida & Entradas  & 72 & Región   & Indonesa \\
        \midrule
        26 & Tipo de comida & Sopas y cremas  & 73 & Región   & Portuguesa \\
        \midrule
        27 & Tipo de comida & Acompañamientos  & 74 & Salud   & Bajas en colesterol \\
        \midrule
        28 & Tipo de comida & Emparedados  & 75 & Salud   & Diabéticos \\
        \midrule
        29 & Tipo de comida & Botana  & 76 & Salud   & Sin lactosa \\
        \midrule
        30 & Tipo de comida & Comida rápida  & 77 & Salud   & Celiacos \\
        \midrule
        31 & Sabor & Dulce  & 78 & Salud   & Alérgicos a los frutos secos \\
        \midrule
        32 & Sabor & Salado  & 79 & Salud   & Colón irritable \\
        \midrule
        33 & Sabor & Ácido  & 80 & Salud   & Bajar de peso \\
        \midrule
        34 & Sabor & Amargo  & 81 & Salud   & Vegetarianos \\
        \midrule
        35 & Sabor & Umami  & 82 & Temperatura & Frío \\
        \midrule
        36 & Sabor & Picante  & 83 & Temperatura & Templado \\
        \midrule
        37 & Ocasión & Halloween  & 84 & Temperatura & Caliente \\
        \midrule
        38 & Ocasión & Navidad  & 85 & Gente & Bebes \\
        \midrule
        39 & Ocasión & San Valentín  & 86 & Gente & Niños \\
        \midrule
        40 & Ocasión & Ocasión especial & 87 & Gente & Adultos \\
        \midrule
        41 & Ocasión & Primavera  & 88 & Gente & Familiares \\
        \midrule
        42 & Ocasión & Verano  & 89 & Gente & Adultos mayores \\
        \midrule
        43 & Ocasión & Otoño  & 90 & Textura & Liquidas \\
        \midrule
        44 & Ocasión & Invierno  & 91 & Textura & Blandas \\
        \midrule
        45 & Ocasión & Desayunos  & 92 & Textura & Semi-blandas \\
        \midrule
        46 & Ocasión & Almuerzos  & 93 & Textura & Crujientes \\
        \midrule
        47 & Ocasión   & Comida & 94 & Textura & Duras \\
        \bottomrule
      \end{tabular}
      \caption{Categorías propuestas para el caso de estudio}
      \label{table:caracteristicas}
    \end{center}
  \end{table}

\begin{landscape}
\section{Diccionario de Datos}
  Los valores mínimos y máximos de los atributos son determinados por la longitud cuando el tipo de dato amerita una validación por longitud, como lo son las cadenas de texto; y una valoración por rango, para aquellos datos donde solo se aceptan datos que pertenezcan a cierto conjunto, como lo es un rango de numeros, denotado por la forma [inicio:fin].
  \begin{table}[h]
    \begin{center}
      \begin{tabular}{| c | c | c | p{3cm} | p{12cm}|} 
        \toprule
        Entidad & Atributo & Tipo & Min / Max & Descripción\\
        \midrule
        Restaurante & Id  & Long & [1:2\^{ }63-1] & Es el identificador de restaurante \\
        \midrule
        Restaurante & Name  & String & Min: 5, Max: 20  & Nombre del restaurante que se registrara en el sistema \\
        \midrule
        Restaurante & Address  & String & Min:10, Max:50  & Dirección del Restaurante \\
        \midrule
        Restaurante & Longitude  & Float & [-180:180]  & Indica la longitud de la ubicación geográfica del restaurante \\
        \midrule
        Restaurante & latitude  & Float & [-90:90]  & Indica la latitud de la posición geográfica del restaurante\\
        \midrule
        Restaurante & averagePrice  & Integer & [1:5]  & Indica un precio aproximado de los platillos del restaurante, donde 1 identifica a un precio menor y 5 a un precio mayor en el restaurante\\
        \midrule
        Restaurante & External url info  & String & Min:20, Max:30  & Dirección url de una fuente de datos       		    externa \\
        \midrule
        User & Birthdate  & Date & [Fecha actual - 100 años:Fecha actual - 13 años]  & Indica la fecha de nacimiento del usuario\\
        \midrule
        User & Email  & String & Min:5, Max:50 & Email del usuario, necesario para la identificación del mismo\\
        \midrule
        User & Password  & String & Min:8, Max:20  & Contraseña del usuario. Necesaria para su identificación.\\
        \midrule
        User & since  & Date & - & Indicará desde que fecha el usuario se dio de alta en el sistema \\
        \bottomrule
      \end{tabular}
    \end{center}
  \end{table}
 \newpage
  \begin{table}
    \begin{center}
      \begin{tabular}{| c | c | c | p{3cm} | p{12cm}|} 
        \toprule
        Entidad & Atributo & Tipo & Min / Max & Descripción\\
        \midrule
        User & Id  & Long & [1:2\^{ }63-1] & Identificador numérico del usuario\\
        \midrule
        User & Name  & String & Min:5, Max:30  & Nombre del usuario \\
        \midrule
        User & Lastname  & String & Min:5, Max:30 & Apellido del usuario \\
        \midrule
        User & Gender  & Char & [F:M]  & Género del usuario \\
        \midrule
        Dish & Id & Long & [1:2\^{ }63-1] & Identificador numérico del platillo\\
        \midrule
        Dish & Name  & String & Min:5, Max:50  & Nombre del Platillo \\
        \midrule
        Dish & Description & String & Min:60, Max:500  & Contendrá las características del platillo, incluyendo la lista de los ingredientes. \\
        \midrule
        Dish & Picture  & String & Min:60, Max:200 & Contendrá la ruta de la imagen del platillo \\
        \midrule
        Category & Id & Long & [1:2\^{ }63-1] & Contendrá el identificador de la categoría que pertenece \\
        \midrule
       	Category & Name  & String & Min:5, Max:20  & Nombre de la categoría \\
        \midrule
        Category & Type  & String & Min:4, Max:15  & A que tipo pertenece dicha categoría,ejemplo:Region \\
        \midrule
        Rated & Rating  & Float & [0.0:5.0]  & ayudara para saber que calificación le dio el usuario 			al platillo recomendado  \\
        \midrule
        Rated & On  & Date & -  & ayudara a saber en que fecha hizo ese rating \\
        \midrule
        Cliked & Last  & Date & -  & La ultima fecha en que visualizo el platillo \\
        \midrule
        Cliked & Times  & Int & [1:2\^{ }31-1]  & Las veces en las que a visualizado ese platillo \\
        \midrule
        Uploaded & On  & Date & - & En que fecha el usuario subió el platillo \\
        \bottomrule
      \end{tabular}
      \caption{Diccionario de datos}
      \label{Diccionario de datos}
    \end{center}
  \end{table}
\end{landscape}  
\newpage
\chapter{Politica de Privacidad}
Bonapettit se compromete a respetar la privacidad la información personal de los
usuarios que acceden al sitio web www.bonapettit.com y en vista de cumplir con las
políticas de seguridad respectivas concernientes a todo sitio web, que deberían ser obligatorias, informo a ustedes lo siguiente.  

\section{Privacidad de los datos personales}
Sus datos personales le corresponden solo a usted y este sitio web es responsable de no revelar ninguna clase de información que le pertenezca (como email, nombre y edad), salvo su expresa autorización o fuerzas de naturaleza mayor de tipo legal que lo involucren, como hackeos o suplantaciones.

\section{Uso de la información que recopila Expansión}
A fin de poder prestarle un amplio abanico de servicios, las categorías de datos que recopilamos son las siguientes:

\subsection{Obtención de su información}
Todos sus datos personales proporcionados en este sitio son suministrados por usted mismo, haciendo uso entero de su libertad. La información aquí almacenada solo comprende datos básicos ingresados mediante formularios de contacto.
\newpage
\subsection{Diversas opciones relativas a los datos personales}
Cada vez que usted accede a un servicio concreto que requiere registro, solicitamos que proporcione datos personales. Si vamos a utilizar estos datos con fines distintos de aquellos para los que se hayan recopilado, le pediremos su consentimiento previo.


Si decidimos utilizar los datos personales con una finalidad distinta de la que se describe en esta Política de Privacidad o en las notificaciones específicas de cada servicio, usted tendrá la posibilidad de manifestar claramente su oposición.
bonappetit no recopilará ni utilizará información confidencial para fines distintos de los establecidos en la presente Política o en los avisos adicionales específicos sin su consentimiento previo.

La mayoría de los navegadores están configurados para aceptar cookies, pero podrá
reconfigurar su navegador para rechazar todas las cookies o para que el sistema le
indique en qué momento se envía una. Sin embargo, si las cookies están inhabilitadas, es posible que algunas características y servicios de Expansión no funcionen de manera adecuada.

\subsection{Seguridad de su información personal}
Este sitio web se hace responsable de velar por su seguridad, por la privacidad de su información y por el respeto a sus datos, de acuerdo con las limitaciones que la actual Internet nos provee, siendo conscientes que no estamos excluidos de sufrir algún ataque por parte de crackers o usuarios malintencionados que ejerzan la delincuencia informática.
\subsection{Estadísticas y otros sitios afines a Bonapettit}
Para el correcto funcionamiento de este sistema, se hacen uso de diversos cookies, tanto del sitio como de nuestros proveedores ,almacenados en su equipo o mediante el uso de algún script.

Las cookies usadas solo se almacenan en su equipo con fines estadísticos, siendo omitidos los datos que pudieran resultar en recopilación de carácter importante.

\subsection{Modificaciones de esta Política de privacidad}
Tenga en cuenta que la presente Política de Privacidad podrá modificarse según se
considere oportuno. No limitaremos sus derechos derivados de la presente política sin su expreso consentimiento. Esperamos que la mayoría de las modificaciones que
realicemos sean menores. Publicaremos las modificaciones en esta página y, si éstas son significativas, le enviaremos el correspondiente aviso destacado, como, por ejemplo, una notificación por correo electrónico (en el caso de determinados servicios). Cada versión de la presente Política de Privacidad incorporará la fecha de entrada en vigor al principio de la página.

\newpage
\chapter{Licencia}
\section{Licencia Apache}

La licencia Apache (Apache License o Apache Software License para versiones anteriores a 2.0) es una licencia de software libre permisiva creada por la Apache Software Foundation 
(ASF).6 La licencia Apache (con versiones 1.0, 1.1 y 2.0) requiere la conservación del aviso de derecho de autor y el descargo de responsabilidad, pero no es una licencia copyleft, ya que no requiere la redistribución del código fuente cuando se distribuyen versiones modificadas.

Al igual que otras licencias de software libre, todo el software producido por la ASF o cualquiera de sus proyectos está desarrollado bajo los términos de esta licencia, es decir, la licencia permite al usuario del software de la libertad de usar el software para cualquier propósito, para distribuirlo, modificarlo y distribuir versiones modificadas del software, bajo los términos de la licencia, sin preocuparse de las regalías.

\section{Condiciones de la licencia}

Como cualquier otra de las licencias de software libre, la Licencia Apache permite al usuario del software la libertad de usarlo para cualquier propósito, distribuirlo, modificarlo, y distribuir versiones modificadas de ese software.

La licencia Apache es permisiva ya que no exige que las obras derivadas (versiones modificadas) del software se distribuyan usando la misma licencia (a diferencia de las licencias copyleft, ni siquiera que se tengan que distribuir como software libre/open source. Todavía requiere la aplicación de la misma licencia a todas las partes no modificadas y en cada archivo de licencia, así como los derechos de autor, patentes, marcas, y las comunicaciones originales de atribución de código redistribuido se deben mantener (con exclusión de avisos que no pertenezcan a ninguna parte de los trabajos derivados); y, en cada cambio de la licencia de archivo, se debe añadir la notificación que indica que se han realizado cambios a ese archivo.


La licencia Apache sólo exige que se mantenga un aviso que informe a los receptores que en la distribución se ha usado código con la licencia Apache. Así, en contraste a las licencias copyleft, quienes reciben versiones modificadas de código con licencia Apache no reciben necesariamente las mismas libertades. O, si se considera la situación desde el punto de vista de los licenciatarios de código con licencia Apache, reciben la libertad de usar el código de la forma que prefieran, incluyendo su uso en productos de código cerrado.\\

Se deben añadir dos archivos en el directorio principal de los paquetes de software redistribuidos:\\\\
\textbf{LICENSE:}
Una copia de la licencia.\\\\
\textbf{NOTICE:}
Un documento de texto, que incluye los avisos obligatorios del software presente en la distribución y una copia legible de estas notificaciones debe ser distribuidas como parte de los trabajos derivados, dentro de la forma de código fuente o documentación, o dentro de una pantalla generada por las obras derivadas (donde aparecen normalmente este tipo de notificaciones a terceros).\\

El contenido del archivo no modifica la licencia, ya que son sólo para fines informativos, y añadiendo más avisos de atribución como adiciones al texto de aviso es admisible, siempre que estos avisos no puedan entenderse como una modificación de la licencia. Las modificaciones pueden tener avisos de copyright adecuados, y pueden proporcionar diferentes términos de licencia para las modificaciones.\\

A menos que se indique expresamente lo contrario, cualquier contribución presentada por un licenciatario a un emisor de licencia estarán bajo los términos de la licencia sin ningún tipo de términos y condiciones, pero esto no impide acuerdos por separado con el licenciante en relación con estas contribuciones.\\

\newpage
\chapter{CONDICIONES DE USO, REPRODUCCIÓN Y DISTRIBUCIÓN}
\section{Definiciones:}
\textbf{Licencia:} hace referencia a las condiciones de uso, reproducción y distribución, según la definición establecida en las Secciones 1 a 9 de este documento.\\\\
\textbf{Licenciador:} hace referencia al propietario de los derechos de autor, o la entidad autorizada por el mismo, que otorga la licencia.\\\\

\textbf{Entidad legal:} hace referencia a la unión de la entidad actuante y todas las demás entidades que la controlan, son controladas por ella o están sujetas a un control común con dicha entidad. Para los fines de esta definición, el control es (i) la potestad directa o indirecta para dirigir dicha entidad, ya sea mediante contrato o de otro modo; o (ii) la titularidad de al menos un cincuenta por ciento  de las acciones, o (iii) la propiedad efectiva de dicha entidad.\\\\
 
Usted (o Su) hace referencia a una persona o entidad legal que ejerza las autorizaciones otorgadas por esta Licencia.\\\\

La forma del Código hace referencia a la forma preferente de realizar modificaciones, como por ejemplo, el código fuente del software, la fuente de la documentación o los archivos de configuración.\\\\

La forma del Objeto hace referencia a cualquier forma resultante de la transformación mecánica o traducción de una forma de Código, como por ejemplo, código de objeto compilado, documentación generada y conversiones a otros tipos de medios.\\\\

\textbf{Obra:} hace referencia a la obra de autor, ya sea en forma de Código o de Objeto, disponible en virtud de esta Licencia, según la indicación de copyright incluida o incorporada a la obra (se facilita un ejemplo en el apéndice más adelante).\\\\

\textbf{Obras derivativas:} hace referencia a todas las obras, tanto en forma de Código como de Objeto, que estén basadas en (o derivadas de) la Obra y en las que el conjunto de las revisiones de modificación, anotaciones, elaboración y otros cambios representan, en total, una obra de autoría original. Para los fines de esta Licencia, las Obras Derivativas no incluirán las obras que sea posible separar o que compartan un simple vínculo (o unión por nombre) con las interfaces de la Obra y de las Obras Derivativas de la misma.\\\\

\textbf{Contribución:} hace referencia a cualquier obra de autor, incluida la versión original de la Obra y todas las modificaciones y adiciones a dicha Obra u Obras Derivativas de la misma, que se envíen deliberadamente al Licenciador para su inclusión en la Obra por el titular de los derechos de autor o por una persona o entidad legal autorizada para ello en representación del titular de los derechos. Para los fines de esta definición, enviar hace referencia a cualquier forma de comunicación electrónica, verbal o escrita efectuada por el Licenciador o sus representantes, como por ejemplo, la comunicación en listas de correo electrónico, sistemas de control de código fuente y sistemas de seguimiento de problemas gestionados por o en representación del Licenciador con el fin de comentar y mejorar la Obra, pero con exclusión de las comunicaciones descritas claramente o designadas por escrito por el titular de los derechos como No contribución.\\\\

\textbf{Contribuyente:} hace referencia al Licenciador o cualquier persona o Entidad Legal en cuyo nombre haya recibido el Licenciador una Contribución que se incorpore posteriormente a la Obra.\\\\

\section{Concesión de licencia de derechos de autor:}De acuerdo con las condiciones de esta Licencia, por la presente, cada Contribuyente le otorga a Usted una licencia de derechos de autor irrevocable, perpetua, internacional, no exclusiva, sin cargas ni regalías, para reproducir, preparar Obras Derivativas, mostrar u operar públicamente, sublicenciar y distribuir la Obra y tales Obras Derivativas en forma de Código o de Objeto.\\\\
\newpage
\section{Concesión de licencia patente.}De acuerdo con las condiciones de esta Licencia, por la presente, cada Contribuyente le otorga a Usted una licencia de patente irrevocable (excepto en los casos indicados en esta sección), perpetua, internacional, no exclusiva, sin cargas ni regalías, para utilizar o hacer utilizar, vender, ofertar, importar y transferir de otro modo la Obra, siendo de aplicación esta Licencia solamente a las solicitudes de patente otorgables por un Contribuyente que afecten exclusivamente a su(s) Contribución(es) o combinación de sus Contribuciones a la Obra a la que se incorporaron. Si usted iniciara un proceso legal contra cualquier entidad (incluidas las contrademandas en un proceso) alegando que la Obra o una Contribución incorporada a la Obra constituye una infracción directa o contributiva de una patente, todas las licencias de patentes otorgadas a Usted en virtud de esta Licencia sobre esa Obra cesarán a partir de la fecha en que se instituya dicho proceso.\\\\
\section{Redistribución.}Usted podrá reproducir y distribuir copias de la Obra o de las Obras Derivativas de la misma en cualquier medio, con o sin modificaciones, en forma de Código o de Objeto, siempre que cumpla las siguientes condiciones:
\begin{itemize}
    \item Deberá facilitar a todos los demás receptores de la Obra u Obras Derivativas una copia de esta Licencia.
    \item Deberá indicar claramente las modificaciones que haya realizado en los archivos.
    \item Deberá conservar en la forma de Código de todas las Obras Derivativas que Usted distribuya todas las advertencias relativas a los derechos de autor, patentes, marcas registradas y atribuciones en la forma de Código de la Obra, con exclusión de las advertencias que no pertenezcan a ninguna parte de las Obras Derivativas.
    \item Si la Obra incluye un archivo de texto de ADVERTENCIA/AVISO como parte de su distribución, todas las Obras Derivativas que Usted distribuya deberán incluir una copia legible de las advertencias de atribución contenidas en dicho archivo de ADVERTENCIA, con exclusión de las advertencias que no pertenezcan a ninguna parte de las Obras Derivativas, en al menos uno de los lugares siguientes: En el contenido del archivo de texto de ADVERTENCIA o AVISO distribuido como parte de la Obra Derivativa; en el contenido de la forma de Código o en la documentación, si estos se facilitan junto con las Obras Derivativas; o mediante un dispositivo de muestra generado por las Obras Derivativas, en el lugar en que normalmente aparezcan tales advertencias externas. El contenido del archivo de ADVERTENCIA es meramente informativo y no modifica la Licencia. Usted podrá añadir Sus propias advertencias de atribución en las Obras Derivativas que Usted distribuya, junto con el texto de ADVERTENCIA de la Obra, o como adición al mismo, siempre que estas advertencias de atribución adicionales no puedan interpretarse como una modificación de la Licencia. Usted podrá añadir Su propia declaración respecto a los derechos de autor en Sus modificaciones y podrá añadir distintas condiciones en la Licencia para el uso, la reproducción o la distribución de Sus modificaciones o de las Obras Derivativas en conjunto, siempre que Su uso, reproducción y distribución de la Obra cumpla por lo demás las condiciones establecidas en esta Licencia.
\end{itemize}
\section{Envío de contribuciones}A menos que Usted indique lo contrario, todas las Contribuciones que Usted envíe deliberadamente al Licenciador para su inclusión en la Obra, estarán sujetas a las condiciones de esta Licencia sin aplicación de condiciones adicionales. No obstante, ninguna disposición de este documento invalidará ni modificará las condiciones de cualquier otro acuerdo de Licencia que usted pueda haber suscrito con el Licenciador en relación con tales Contribuciones.
\section{Marcas comerciales.}Esta Licencia no autoriza el uso de nombres comerciales, marcas comerciales, marcas de servicios o nombres de productos del Licenciador, excepto cuando lo requiera el uso razonable y habitual en la descripción del origen de la Obra y la reproducción del contenido del archivo de ADVERTENCIA.
\section{Exención de garantía.}A menos que lo exijan las leyes pertinentes o se acuerde por escrito, el Licenciador ofrece la Obra (y cada Contribuyente ofrece sus Contribuciones) TAL CUAL, SIN GARANTÍAS NI CONDICIONES DE NINGÚN TIPO, ya sean expresas o implícitas, como por ejemplo, cualquier garantía o condición sobre TÍTULO, NO INFRACCIÓN, APTITUD PARA EL COMERCIO o IDONEIDAD PARA UN FIN PARTICULAR. Usted es el/la único(a) responsable de determinar si es apropiado utilizar o redistribuir la Obra y asume todos los riesgos asociados a Su ejercicio de los permisos otorgados en esta Licencia.
\section{Responsabilidad limitada.}Bajo ninguna circunstancia ni fundamento legal, sea por ilícito civil extracontractual (incluida la negligencia), por contrato o de otro modo, a menos que lo exijan las leyes pertinentes (como en el caso de actos de negligencia deliberados y graves) o se haya acordado por escrito, será responsable ningún Contribuyente ante Usted por daños de ningún tipo, ya sean directos, indirectos, especiales, incidentales o consecuentes ocasionados como resultado de esta Licencia o por el uso o imposibilidad de uso de la Obra (como por ejemplo los daños por pérdida de clientes, pérdida de actividad, avería o mal funcionamiento de los ordenadores o cualquier otra forma de perjuicios o pérdidas comerciales), incluso si dicho Contribuyente hubiese sido advertido de la posibilidad de tales perjuicios.
 \newpage
 \section{ Aceptación de garantías o responsabilidad adicional}En Su redistribución de la Obra o de las Obras Derivativas de la misma, Usted podrá ofrecer y cobrar por la aceptación de asistencia, garantías, indemnización u otras obligaciones de responsabilidad y/o derechos en virtud de esta Licencia. No obstante, al aceptar tales obligaciones, Usted podrá actuar solamente en Su propio nombre y bajo Su propia responsabilidad, no en nombre de ningún otro Contribuyente, y solamente si Usted accede a indemnizar, defender y eximir a cada Contribuyente de cualquier tipo de responsabilidad o disputas contra dicho Contribuyente, como resultado de u aceptación de tales garantías o responsabilidades adicionales.\\\\

FIN DE LAS CONDICIONES\\\\

Derechos de autor 2016 [López Garduño Blanca Azucena Bautista de Jesús Héctor \\Gerardo
Castro Esparza José Antonio]\\

   Autorizado en virtud de la Licencia de Apache, Versión 2.0 ; se prohíbe utilizar este archivo excepto en cumplimiento de la Licencia.
   Podrá obtener una copia de la Licencia en:\\

       http://www.apache.org/licenses/LICENSE-2.0\\

   A menos que lo exijan las leyes pertinentes o se haya establecido por escrito, el software distribuido en virtud de la Licencia se distribuye “TAL CUAL”, SIN GARANTÍAS NI CONDICIONES DE NINGÚN TIPO, ya sean expresas o implícitas.
   Véase la Licencia para consultar el texto específico relativo a los permisos y limitaciones establecidos en la Licencia.

