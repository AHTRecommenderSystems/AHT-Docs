\chapter {Antecedentes}
\section {Sistemas de recomendación}
  \paragraph {Un sistema de recomendaci\'on es un sistema el cual proporciona a los usuarios una conjunto de sugerencias, dichas sugerencias son personalizadas las cuales podemos catalogarlas como recomendaciones, dichas recomendaciones son sobre un determinado tipo de los cuales son llamados items. Los sistemas de recomendaci\'on estudian las caracter\'isticas de cada usuario y mediante un procesamiento de los datos, encuentra un subconjunto de items que pueden resultar de inter\'es para el usuario. Clasificaci\'on Los sistemas de recomendaci\'on (SR) se clasifican en 3 tipos: SR con Filtrado basado en Contenido: Las recomendaciones est\'an basadas en el conocimiento que se tiene sobre los items que el usuario a proporcionado, pueden ser de forma impl\'icita o expl\'icita, y se le recomendar\'an items similares que le puedan gustar o interesar. SR basado en Filtrado Colaborativo: El filtrado colaborativo consiste en ver que usuarios son similares al usuario activo (o usuario al que hay que realizarle las recomendaciones) y despu\'es recomendar aquellos items que no han sido votados por el usuario activo y que han resultado bien valorados por los usuarios similares. SR con m\'etodos de Filtrado H\'ibrido: Mezclan alguno de los dos filtrados mencionados anteriormente para realizar recomendaciones e incluso lo combinan con alguna otra t\'ecnica de inteligencia artificial como pueda ser la l\'ogica borrosa o la computaci\'on evolutiva.}

\section {Ejemplos de sistemas de recomendación}
  \subsection {Spotify}
    \paragraph {Permite escuchar en modo radio buscando por artista, álbum o listas de reproducción creadas por los propios usuarios. Spotify no es una api, pero hace uso de sistemas de recomendación. Componentes:Transferencia de archivos de audio por internert a través de la combinación de servidor basado en el streaming y en la transferencia peer-to-peer (P2P). Plataformas:Multiplataforma Año: 2008, 2009. Trabajo Terminal Sistema Generador de Recomendaciones para una Tienda En-Línea de Videojuegos.\cite{5} Sistema de información para la venta de artículos de entretenimiento. Dicho Trabajo Terminal no es una api pero es un sistema de recomendaciones. Componentes:Recomendación, TopN, mercancía, keyword, pedido Plataforma:web. año:2010}

  \subsection{PredictionIO}
    \paragraph{Proporciona los recursos necesarios para crear un servidor de recomendaciones usando machine learning. Todo a través de una API REST que se comunica con las distintas aplicaciones clientes y va recolectando datos para aplicar los más de 20 algoritmos de recomendaciones precargados. Componentes:Cuenta con varios SDK para integrarlo en sus aplicaciones como Java, Python, Ruby o PHP.Ademas de sistemas de recomendación y aprendizaje máquina Plataformas:Linux/MacOSX, Vagrant(VirtualBox), Terminal.com Snap (web) ,Source Code.\cite{6}}

  \subsection{Easyrec}
    \paragraph{Sistema de personalización de propósito general, utiliza una API de servicio web actual está personalizada para proporcionar tiendas online con recomendaciones de ítems. Cabe mencionar que es una aplicación lista para usar, no un framework algorítmico. No es una API, pero si hace eso de una API el cual hace las recomendaciones. Componentes: aplicación para almacenar servicios de recomendación, API para varias interfaces de servicio web. Año:7 de mayo 2013\cite{7}}

    \subsubsection{REST API}
      \paragraph{API easyrec es capaz de recibir acciones desde o enviar recomendaciones a la aplicación web en estilo REST. Todas las acciones de los usuarios enviados a easyrec son analizados por el ruleminer, que se ejecuta una vez al día de forma predeterminada. Si similitudes significativas en el comportamiento del usuario se pueden encontrar, se generan recomendaciones y son accesibles al instante utilizando los servicios web de recomendación. Componentes: XML o notación JSON para mostrar las recomendaciones. \cite{8}}

  \subsection{LensKit}
    \paragraph{Desarrollado por GroupLens en la Universidad de Minessota, es una API para desarrollo de sistemas de recomendación, que funciona bajo algoritmos k-nn y similitud del coseno para realizar las recomendaciones de los artículos. Su implementación puede ser observada en el sistema de recomendación de películas MovieLens, igualmente desarrollado por el mismo equipo de trabajo.}
    
