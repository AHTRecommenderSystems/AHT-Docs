\section{Metodología}
	\subsection{Descripción}
	  Para ese proyecto se plantea usar una metodología de prototipado evolutivo, que se caracteriza por que en su modelo de trabajo un prototipo es construido, probado y finalmente reconstruido las veces que sea necesario hasta que un prototipo aceptable es finalmente alcanzado del cual el sistema completo o producto puede ser totalmente desarrollado.

	  Este modelo funciona bien en escenarios donde no se conocen por completo los requerimientos. Así los prototipos, modelos de software con una funcionalidad limitada, permiten al usuario evaluar los propósitos del desarrollador y probarlos antes de su implementación. También ayuda a entender los requerimientos específicos del usuario y que no pudieron haber sido considerados por el desarrollador durante el diseño del sistema.

	\subsection{Prototipos esperados}
	  De acuerdo al modelo de desarrollo elegido, se han establecido los siguientes prototipos y sus alcances que se pueden observar en el cuadro~\ref{table:prototipos}

	  \begin{table}[h]
	\begin{center}
	  \begin{tabular}{ | c | c | }
	    \toprule
	    	\textbf{Versión} & \textbf{Alcance} \\
	    \midrule
	    	Prototipo 1 & Obtener y abstraer los datos a utilizar\\
	    \midrule
	    	Prototipo 2 & Visualizar los datos en recomendaciones por contenido\\
	    \midrule
	    	Prototipo 3 & Obtener recomendaciones para diversos fines (sistemas híbridos) \\
	    \midrule
	    	Prototipo 4 & Desarrollar un sistema híbrido de recomendación para platillos y restaurantes. \\
	    \bottomrule
		\end{tabular}
	  \caption{Cuadro de prototipos esperados}
  	  \label{table:prototipos}
  	\end{center}
	\end{table}

