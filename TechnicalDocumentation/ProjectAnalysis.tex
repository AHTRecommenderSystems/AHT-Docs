\chapter{Análisis general del proyecto}
	Para el desarrollo del sistema en su totalidad, se realizó el siguiente análisis general que define en su totalidad al proyecto.

	\section{Características}
		Para que el sistema se considere que ha cumplido con los objetivos planteados debe contar con las siguientes características dentro de su funcionalidad.
		\begin{itemize}			
			\item El sistema permitirá la búsqueda de platillos.
			\item El sistema permitirá obtener recomendaciones de acuerdo a las características de un platillo.
			\item El sistema obtendrá información de la interacción del usuario a través de evaluaciones explícitas, y conteo de clics implícitos hacia los platillos para obtener recomendaciones personalizadas para dicho usuario.
			\item El sistema permitirá el registro de usuarios finales.
			\item El sistema generará recomendaciones de platillos de acuerdo a la información proporcionada de los usuarios registrados, con base en sus evaluaciones y características.
			\item El sistema permitirá al usuario agregar nuevos platillos que haya consumido en cierto restaurante.
			\item El sistema permitirá al usuario administrar los platillos que agregó. 
		\end{itemize}

	\section{Restricciones}
		Debido a diferentes aspectos, el sistema contará con las siguientes restricciones.
		\begin{itemize}
			\item El sistema se verá limitado a las tecnologías empleadas para su desarrollo.
			\item El sistema puede verse limitado debido a la dependencia de fuentes de información externas.
			\item El sistema puede verse limitado en desempeño y precisión debido a la cantidad de información almacenada y a la complejidad del problema a resolver.
		\end{itemize} 

	\section{Estudio de factibilidad}
Después de definir la problemática presente y presentar la propuesta de solución, es  pertinente realizar un estudio de factibilidad para determinar la infraestructura tecnológica y la capacidad técnica que implica la implementación de la API en cuestión. Este análisis permite determinar las posibilidades de diseñar a API propuesta y su puesta en marcha. 
\\\\
A continuación se describen los aspectos que se toman en cuenta para este análisis.
\subsection{Factibilidad técnica }
Consiste en realizar una evaluación de la tecnología existente; este estudio está destinado a recolectar información sobre los componentes tecnológicos que posee este equipo de desarrollo y la posibilidad de utilizarlos en el desarrollo e implementación de este trabajo y de ser necesario, los requerimientos tecnológicos que deben ser adquiridos su desarrollo e implementación. 
\\\\
Hardware 
\\\\
Para el desarrollo de la API se cuentan con 3 computadoras personales, las cuales cuentan 
con las siguientes características:
\begin{itemize}
 \item MacBook Pro 13: 4 GB DDR3 RAM, Procesador Intel i5 a 2.5 GHz. S.O
 \item HP  8 GB DDR3 RAM, Procesador A10 2.1 GHz. S.O. Ubuntu 14.04.
 \item Acer 8 GB DDR3 RAM Procesador AMD A6 
\end{itemize}


Los módulos desarrollados para la primera parte del Trabajo Terminal serán expuestos de 
manera local, accediendo a ellos a través de una red local. Para la segunda entrega del 
Trabajo Terminal se tiene planeado exponerlos en un servidor, para lo cual se contará un 
servicio de hosting. No se incluirá en este documento la opción elegida para el hosting ni el 
costo del mismo, ya que no se sabe con exactitud cual se elegirá, a pesar de que se han 
tomado en cuenta varios, no se ha llegado a una decisió.
\newpage
Software  
\begin{itemize}
\item Java
\item Neo4j
\item Hibernate
\item JavaScript
\item HTML
\item Boostrap
\item Spring
\item Maven
\end{itemize}
Estas tecnologías son necesarias para el desarrollo de esta API. Cada una cumple con un  objetivo específico para las cuales fueron utilizadas, pero no son indispensables. No son indispensables, ya que se encuentran muchos otros lenguajes y frameworks con los cuales se pueden reemplazar estas tecnologías. 
\\\\
Costos 
\\\\
Los costos que generará el desarrollo de este framework se calcularon de la siguiente manera: 
\begin{itemize}
\item El manejo roles se repartió en el equipo, es decir, todos tuvieron que analizar, 
diseñar y desarrollar.  
\item Nos basamos en un sueldo promedio al cual aspiran estudiantes de la carrera de 
Ing. en Sistemas Computacionales, que es de \$ 80
\end{itemize}
Después de aclarar lo siguiente, los costos operacionales (mano de obra) se calcularon así:
\begin{itemize}
\item 3 personas que desarrollan diferentes actividades 
\item Cada uno gana \$80 la hora desarrollando la API 
\item Se toma en cuenta que se  trabajan los 7 días de la semana 4 horas cada uno de ellos, así: 
\end{itemize}
(\$80) * (4 hrs) * (250 dias) = \$80,000 
 
El precio estimado de este sistema es de \$80000 mx solamente tomando en cuenta la manos de obra, sin contemplar nuevos equipos, reuniones, transportes, comida y horas extras. 
\newpage
	\section{Metodología}
\subsection{Descripción}
Para ese proyecto se plantea usar una metodología de prototipado evolutivo, que se caracteriza por que en su modelo de trabajo un prototipo es construido, probado y finalmente reconstruido las veces que sea necesario hasta que un prototipo aceptable es finalmente alcanzado del cual el sistema completo o producto puede ser totalmente desarrollado.

Este modelo funciona bien en escenarios donde no se conocen por completo los requerimientos. Así los prototipos, modelos de software con una funcionalidad limitada, permiten al usuario evaluar los propósitos del desarrollador y probarlos antes de su implementación. También ayuda a entender los requerimientos específicos del usuario y que no pudieron haber sido considerados por el desarrollador durante el diseño del sistema.

\subsection{Prototipos esperados}
De acuerdo al modelo de desarrollo elegido, se han establecido los siguientes prototipos y sus alcances que se pueden observar en el cuadro~\ref{table:prototipos}

\begin{table}[h]
	\begin{center}
		\begin{tabular}{ | c | c | }
			\toprule
			\textbf{Versión} & \textbf{Alcance}                                                                 \\
			\midrule
			Prototipo 1       & Obtener y abstraer los datos a utilizar                                          \\
			\midrule
			Prototipo 2       & Visualizar los datos en recomendaciones por contenido                            \\
			\midrule
			Prototipo 3       & Obtener recomendaciones para diversos fines (sistemas híbridos)                 \\
			\midrule
			Prototipo 4       & Desarrollar un sistema híbrido de recomendación para platillos y restaurantes. \\
			\bottomrule
		\end{tabular}
		\caption{Cuadro de prototipos esperados}
		\label{table:prototipos}
	\end{center}
\end{table}


	\section {Descripción y módulos del sistema}

  \subsection{Arquitectura general}
    Para desarrollar un sistema de recomendación se han planteado los siguientes módulos funcionales de la API, el cual será utilizado por el desarrollador final para que, en conjunto con su aplicación final realice la integración de las funciones disponibles en el API junto al sistema de recomendación final. Esto se puede denotar en el siguiente diagrama.

\newpage
    \begin{landscape}
      \begin{figure}[h!]
      \centering
      \includegraphics[width=22.5cm,height=12cm]{./images/architecture.png}
      \caption{Diagrama general del sistema}
    \end{figure}
    \end{landscape}
  \newpage

Cómo se puede apreciar en el diagrama, el sistema se encuentra dividido en tres módulos básicos además de la aplicación final que hace uso de los módulos de la API.
    \begin{itemize}
    \item Módulo de obtención datos
    \item Módulo de evaluación
    \item Módulo de presentación de resultados
    \item Aplicación cliente
  \end{itemize}

\subsection{Módulos de la API}
  \subsubsection{Módulo de obtención de datos}
    Este módulo es el encargado de la conexión a una fuente de datos de la cuál obtendrá los datos a utilizar, mapeándolos dentro de una estandarización propuesta por el equipo de trabajo. Esta obtención permitirá tener la funcionalidad del sistema de recomendación de manera adecuada. 
    Se presenta como un módulo conformado por diferentes interfaces de conexión para fuentes de datos, como lo es un gestor de base de datos. Tiene una interacción directa con el módulo de operaciones lógicas dentro del sistema.
    Dentro de sus características podemos denotar:
    \begin{itemize}
      \item Permite la conexión a una fuente de datos.
      \item A través de interfaces, permite mapear los datos utilizados en cada caso de estudio para el correcto funcionamiento del módulo de operaciones lógicas.
      \item Se encuentra restringido al modelo de datos mínimo propuesto por el equipo de trabajo.
    \end{itemize}

  \subsubsection{Módulo de operaciones lógicas}
    El módulo de operaciones lógicas permitirá, haciendo uso de la información almacenada, obtener evaluaciones de los diferentes artículos de acuerdo a diferentes clasificaciones con base en algoritmos de recomendación que definen los principales tipos de recomendación existentes: basados en contenido, colaborativos e híbridos. Estas evaluaciones deberán ser utilizadas por el módulo de presentación de resultados para devolver predicciones o recomendaciones que estén dentro de los rangos de evaluación propuestos por el desarrollador para su caso de estudio. Cabe destacar que este tipo de algoritmos devolverán evaluaciones cuantitativas, y la representación de estos se definirá dependiendo el caso, siendo lo ideal que a un mayor valor corresponda una mayor similitud o preferencia. Dentro de sus características principales encontramos:
    \begin{itemize}
      \item Hace uso de la conexión proporcionada por el módulo de datos y los parámetros indicados en el módulo de presentación de resultados para devolver las evaluaciones.
      \item Las evaluaciones son cuantitativas, denotando idealmente una mayor relevancia para el usuario como un mayor valor.
      \item El algoritmo de evaluación a utilizar será determinado en la interacción entre éste módulo y el módulo de presentación de resultados.
    \end{itemize}

  \subsubsection{Módulo de presentación}
    Haciendo uso del módulo de operaciones lógicas, este módulo pretende interactuar con la aplicación final para brindar las funciones de recomendación y predicción para cada usuario o artículo. 

\subsection{Aplicación final}
  \subsubsection{Definición}
    En este caso, la aplicación final hará uso de las funciones proporcionadas por los distintos módulos que conforman la API para obtener recomendaciones de los datos que pertenezcan a su caso de estudio. Interactúa directamente con el módulo de abstracción de datos y con el módulo de presentación para hacer uso de la funcionalidad permitida por el API. En este caso, la aplicación final se verá reflejada en un sistema web que permita denotar la funcionalidad de la API para un conjunto de datos de platillos y restaurantes.
	\section{Análisis y gestión de riesgos}
  \subsection{Definición y clasificación}
    \paragraph{El riesgo siempre implica una incertidumbre y una pérdida potencial, al identificar estos riesgos podemos determinar su naturaleza en tres tipos diferentes para este proyecto:}
    \begin{itemize}
      \item Riesgos del proyecto
      \item Riesgos técnicos
      \item Riesgos de negocio
    \end{itemize}
    \paragraph{Así mismo, es necesario clasificar los riesgos existentes de acuerdo  a la probabilidad de que éstos ocurran. Para lo cual se utilizará los siguientes valores por convención.}
    \begin{itemize}
      \item Muy bajo ( < 10\% )
      \item Bajo ( 10 - 25\% )
      \item Moderado (25 - 50\% )
      \item Alto (50 - 75\% )
      \item Muy Alto ( > 75\% )
    \end{itemize}
    \paragraph{Así mismo, todos los riegos depen categorizarse deacuero al impacto que pueden causar en el sistema en las siguientes categorías: Insignificante, Tolerable, Serio, Catastrófico; los planes de contingencia suelen ser desarrollados para aquellos riegos con probabilidad de moderada a muy alta y con un impacto serio o castastrófico.}
    \paragraph{A continuación se plantean los riesgos identificados para el sistema general a lo largo del desarrollo del mismo y sus respectivos planes de acción.}
    \newpage
    \begin{table}[b!]
    \centering
      \begin{tabular}{|p{3cm}|lllll}
        \hline
        \multicolumn{5}{|c|}{{\bf Tabla de Riesgos}} \\ 
        \hline
          \multicolumn{1}{|p{3cm}|}{{\bf Descripcion}} & 
          \multicolumn{1}{p{2cm}|}{{\bf Tipo de Riesgo}} & 
          \multicolumn{1}{p{2cm}|}{{\bf Valoración}} & 
          \multicolumn{1}{p{2cm}|}{{\bf Porcentaje}} & 
          \multicolumn{1}{p{5cm}|}{{\bf Plan de acción}} \\ 
        \hline
          \multicolumn{1}{|p{3cm}|}{Falta de presupuesto} & 
          \multicolumn{1}{p{2cm}|}{Proyecto} & 
          \multicolumn{1}{p{2cm}|}{Serio} & 
          \multicolumn{1}{p{2cm}|}{25\%} & 
          \multicolumn{1}{p{5cm}|}{Buscar un proceso de incubación en empresas como Apache, Eclipse y migrar la plataforma a servicios de hosting gratuitos como Heroku u Openshift.} \\ 
        \hline
          \multicolumn{1}{|p{3cm}|}{Falta por razones personales de miembros del equipo} & 
          \multicolumn{1}{p{2cm}|}{Proyecto} &
          \multicolumn{1}{p{2cm}|}{Serio} & 
          \multicolumn{1}{p{2cm}|}{25\%} & 
          \multicolumn{1}{p{5cm}|}{Rediseñar y adaptar las tareas con nuevos integrantes y habilitar forma de trabajo remota considerando fines de semana.} \\ 
        \hline
          \multicolumn{1}{|p{3cm}|}{Necesidad de escalabilidad de la plataforma} & 
          \multicolumn{1}{p{2cm}|}{Técnico} & 
          \multicolumn{1}{p{2cm}|}{Serio} & 
          \multicolumn{1}{p{2cm}|}{10\%} & 
          \multicolumn{1}{p{5cm}|}{Definir el tipo de escalabilidad a usar dependiendo los recursos monetarios existentes.} \\ 
        \hline
          \multicolumn{1}{|p{3cm}|}{Ataques de Denegación de Servicios (DoS)} & 
          \multicolumn{1}{p{2cm}|}{Técnico} & 
          \multicolumn{1}{p{2cm}|}{Tolerable} & 
          \multicolumn{1}{p{2cm}|}{10\%} & 
          \multicolumn{1}{p{5cm}|}{} \\ 
        \hline
      \end{tabular}
      \caption{Analisis de Riesgos}
      \label{Analisis de riesgos}
    \end{table}
    \clearpage
	\chapter{Tecnologías}
	\paragraph{El uso de alguna tecnología en particular, puede coadyuvar o entorpecer el desarrollo de un sistema, por esto es necesario considerar varios puntos a tratar para seleccionar el conjunto de tecnologías a utilizar en un proyecto, entre estas puede ser la facilidad de uso, la curva de aprendizaje, la funcionalidad y sobretodo el conocimiento existente sobre la tecnología. Para este caso se propone utilizar las siguientes tecnologías.}
	
	\begin{itemize}
	  \item Neo4j
	  \item Java
	  \item Git
	  \item Maven
	\end{itemize}
