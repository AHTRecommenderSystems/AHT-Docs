\begin{thebibliography}{1}

  \bibitem{1}
    Roger Loaiza. ¿Qué es la inteligencia artificial?. “De la información a la informática”. Obtenido desde \url{http://bvs.sld.cu/revistas/san/vol2_2_98/san15298.htm} el 1 de mayo de 2015.
  \bibitem{2}
    Enrique Herrera-Viedma, Sistemas de recomendaciones: herramientas para el filtrado de información en Internet, 2004. Obtenido desde \url{http://www.upf.edu/hipertextnet/numero-2/recomendacion.html} el 2 de mayo de 2015. 
  \bibitem{3}
     Marcos Merino. ¿Qué es una API y para qué sirve? Obtenido desde: \url{http://www.ticbeat.com/tecnologias/que-es-una-api-para-que-sirve/} el 4 de mayo de 2015. 
  \bibitem{4}
     Spotify. Obtenido desde: \url{https://www.spotify.com/mx/about-us/contact/}
  \bibitem{5}
     Sistema Generador de Recomendaciones para una Tienda En-Línea de Videojuegos, 2010. ESCOM IPN. 
  \bibitem{6}
    Gregory Piatetsky. Prediction.io open source machine learning server, 10 de abril de 2014. Obtenido desde: \url{http://www.kdnuggets.com/2014/04/prediction-io-open-source-machine-learning-server.html} el 6 de mago de 2015. 
  \bibitem{7}
    Easyrec, Obtenido desde \url{http://easyrec.org/} el 6 de mayo de 2015 
  \bibitem{8}
    REST API. Obtenido desde: \url{http://easyrec.sourceforge.net/wiki/index.php?title=REST_API_v0.98} el 6 de mayo de 2015. 
  \bibitem{9}
    Netflix. Obtenido desde: \url{https://help.netflix.com/es/node/412} 6 de mayo de 2015.
  \bibitem{10}
    Gediminas Adomavicius and Alexander Tuzhilin \i{Towards the Next Generation of Recommender Systems: A Survey of the State-of-the-Art and Possible Extensions} Obtenido desde: \url{http://homepages.dcc.ufmg.br/~nivio/cursos/ri13/sources/recommender-systems-survey-2005.pdf} 6 de mayo de 2015.
  \bibitem{11}
    Lenskit: Open-Source Tools for Recommender Systems. Obtendo desde \url{http://lenskit.org/}
 \bibitem{12}
    Bases de Datos Orientadas a Grafos. Obtendo desde \url{http://www.bigdatahispano.org/noticias/bases-de-datos-orientadas-a-grafos/}	
 \bibitem{13}
    Bases de Datos Orientadas a Grafos-Caracteristicas. Obtendo desde \url{http://www.bigdatahispano.org/noticias/bases-de-datos-orientadas-a-grafos/}	
\bibitem{14}
    Bases de Datos Orientadas a Grafos-Ventajas. Obtendo desde \url{http://www.bigdatahispano.org/noticias/bases-de-datos-orientadas-a-grafos/}	
    \bibitem{15}
    Bases de Datos Orientadas a Grafos-Desventajas. Obtendo desde \url{http://www.bigdatahispano.org/noticias/bases-de-datos-orientadas-a-grafos/}
    \bibitem{16}
    Neo4j Obtendo desde \url{http://xurxodeveloper.blogspot.mx/2014/03/neo4j-una-base-de-datos-nosql-orientada.html/}
    \bibitem{17}
    Tipos de bases de datos NoSQL Obtendo desde \url{http://xurxodeveloper.blogspot.mx/2014/03/neo4j-una-base-de-datos-nosql-orientada.html/}	
    \bibitem{18}
    Cómo Funciona Neo4j Obtendo desde \url{http://bbvaopen4u.com/es/actualidad/neo4j-que-es-y-para-que-sirve-una-base-de-datos-orientada-grafos/}
    \bibitem{19}
    Características de Neo4j Obtendo desde \url{http://xurxodeveloper.blogspot.mx/2014/03/neo4j-una-base-de-datos-nosql-orientada.html/}	
    \bibitem{20}
    Procesamiento en grafo de forma nativa Obtendo desde \url{http://xurxodeveloper.blogspot.mx/2014/03/neo4j-una-base-de-datos-nosql-orientada.html/}	
    \bibitem{21}
    Almacenamiento en grafo de forma nativa Obtendo desde \url{http://xurxodeveloper.blogspot.mx/2014/03/neo4j-una-base-de-datos-nosql-orientada.html/}	
    \bibitem{22}
    Neo4j-Rendimiento Obtendo desde \url{http://bbvaopen4u.com/es/actualidad/neo4j-que-es-y-para-que-sirve-una-base-de-datos-orientada-grafos/}	
    \bibitem{23}
   	 Ne4j-Agilidad Obtendo desde \url{http://xurxodeveloper.blogspot.mx/2014/03/neo4j-una-base-de-datos-nosql-orientada.html/}	
    \bibitem{24}
    Neo4j-Flexibilidad y escalabilidad Obtendo desde \url{http://xurxodeveloper.blogspot.mx/2014/03/neo4j-una-base-de-datos-nosql-orientada.html/}
    \bibitem{25}
    Casos de Uso de Neo4j Obtendo desde \url{http://xurxodeveloper.blogspot.mx/2014/03/neo4j-una-base-de-datos-nosql-orientada.html/}	
    \bibitem{26}
    Oracle y Java, características. Obtenido desde \url{http://www.oracle.com/es/technologies/java/features/index.html}
     \bibitem{27}
   Tipos de API. Obtenido desde \url{http://www-01.ibm.com/support/knowledgecenter/SS6PEW_9.4.0/com.ibm.help.custom.apis.doc/API_API_Types.html?lang=es}
   	\bibitem{28}
   Bases de datos relacionales. Obtenido desde J. J. King, «QUIST: A System for Semantic Query Optimization in Relational Data Bases», Proc. of the International Conf. on Very Large Databases (1981),
páginas 510-517
   \bibitem{29}
   Bases de datos no relacionales. Obtenido desde \url{ «NoSQL Relational Database Management System: Home Page». Strozzi.it. 2 de octubre de 2007. Consultado el 10 de Noviembre de 2015.}
   \bibitem{30}
   Bases de datos orientadas a grafos. Obtenido desde \url{ http://graphbase.net/}
   \bibitem{31}
   Bases de datos orientadas a grafos. Obtenido desde \url{ http://xurxodeveloper.blogspot.mx/2014/03/neo4j-una-base-de-datos-nosql-orientada.html}
   \bibitem{32}
   InfiniteGraph. Obtenido desde  Joyce Wells (June 26, 2013). "DBTA 100: The Companies That Matter Most in Data". Database Trends and Applications. Retrieved September 8, 2014.
   \bibitem{33}
   Infogrid. Obtenido desde  \url{http://infogrid.org/trac/}
   \bibitem{34}
   Bootstrap. Obtenido desde  \url{http://getbootstrap.com/2.3.2/}
   \bibitem{35}
   Foundation. Obtenido desde  \url{http://foundation.zurb.com/}
   \bibitem{36}
   Maven. Obtenido desde  \url{http://chuwiki.chuidiang.org/index.php?title=Categor%C3%ADa:Maven}
   \bibitem{37}
   Ant. Obtenido desde  \url{http://ant.apache.org/}
   \bibitem{38}
   Git. Obtenido desde  \url{https://git-scm.com/}
   \bibitem{39}
   Subversion \url{http://svnbook.red-bean.com/nightly/es/svn-ch-1-sect-1.html}
   \bibitem{40}
   Java \url{http://www.oracle.com/technetwork/java/index.html}
\end{thebibliography}


\addcontentsline{toc}{chapter}{Bibliografía}
