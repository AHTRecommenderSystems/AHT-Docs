\section{Análisis y gestión de riesgos}
  \subsection{Definición y clasificación}
    \paragraph{El riesgo siempre implica una incertidumbre y una pérdida potencial, al identificar estos riesgos podemos determinar su naturaleza en tres tipos diferentes para este proyecto:}
    \begin{itemize}
      \item Riesgos del proyecto
      \item Riesgos técnicos
      \item Riesgos de negocio
    \end{itemize}
    \paragraph{Así mismo, es necesario clasificar los riesgos existentes de acuerdo  a la probabilidad de que éstos ocurran. Para lo cual se utilizará los siguientes valores por convención.}
    \begin{itemize}
      \item Muy bajo ( < 10\% )
      \item Bajo ( 10 - 25\% )
      \item Moderado (25 - 50\% )
      \item Alto (50 - 75\% )
      \item Muy Alto ( > 75\% )
    \end{itemize}
    \paragraph{Así mismo, todos los riegos depen categorizarse deacuero al impacto que pueden causar en el sistema en las siguientes categorías: Insignificante, Tolerable, Serio, Catastrófico; los planes de contingencia suelen ser desarrollados para aquellos riegos con probabilidad de moderada a muy alta y con un impacto serio o castastrófico.}
    \paragraph{A continuación se plantean los riesgos identificados para el sistema general a lo largo del desarrollo del mismo y sus respectivos planes de acción.}
    \newpage
    \begin{table}[b!]
    \centering
      \begin{tabular}{|p{3cm}|lllll}
        \hline
        \multicolumn{5}{|c|}{{\bf Tabla de Riesgos}} \\ 
        \hline
          \multicolumn{1}{|p{3cm}|}{{\bf Descripcion}} & 
          \multicolumn{1}{p{2cm}|}{{\bf Tipo de Riesgo}} & 
          \multicolumn{1}{p{2cm}|}{{\bf Valoración}} & 
          \multicolumn{1}{p{2cm}|}{{\bf Porcentaje}} & 
          \multicolumn{1}{p{5cm}|}{{\bf Plan de acción}} \\ 
        \hline
          \multicolumn{1}{|p{3cm}|}{Falta de presupuesto} & 
          \multicolumn{1}{p{2cm}|}{Proyecto} & 
          \multicolumn{1}{p{2cm}|}{Serio} & 
          \multicolumn{1}{p{2cm}|}{25\%} & 
          \multicolumn{1}{p{5cm}|}{Buscar un proceso de incubación en empresas como Apache, Eclipse y migrar la plataforma a servicios de hosting gratuitos como Heroku u Openshift.} \\ 
        \hline
          \multicolumn{1}{|p{3cm}|}{Falta por razones personales de miembros del equipo} & 
          \multicolumn{1}{p{2cm}|}{Proyecto} &
          \multicolumn{1}{p{2cm}|}{Serio} & 
          \multicolumn{1}{p{2cm}|}{25\%} & 
          \multicolumn{1}{p{5cm}|}{Rediseñar y adaptar las tareas con nuevos integrantes y habilitar forma de trabajo remota considerando fines de semana.} \\ 
        \hline
          \multicolumn{1}{|p{3cm}|}{Necesidad de escalabilidad de la plataforma} & 
          \multicolumn{1}{p{2cm}|}{Técnico} & 
          \multicolumn{1}{p{2cm}|}{Serio} & 
          \multicolumn{1}{p{2cm}|}{10\%} & 
          \multicolumn{1}{p{5cm}|}{Definir el tipo de escalabilidad a usar dependiendo los recursos monetarios existentes.} \\ 
        \hline
          \multicolumn{1}{|p{3cm}|}{Ataques de Denegación de Servicios (DoS)} & 
          \multicolumn{1}{p{2cm}|}{Técnico} & 
          \multicolumn{1}{p{2cm}|}{Tolerable} & 
          \multicolumn{1}{p{2cm}|}{10\%} & 
          \multicolumn{1}{p{5cm}|}{} \\ 
        \hline
      \end{tabular}
      \caption{Analisis de Riesgos}
      \label{Analisis de riesgos}
    \end{table}
    \clearpage