%%  COVER
\pagenumbering{Alph}
\begin{titlepage}
    \begin{center}
    \begin{tabular}{r c l}
    \includegraphics[scale=.20]{images/ipn} & \textbf{INSTITUTO POLIT\'ECNICO NACIONAL} & \includegraphics[scale=.20]{images/escom}\\ 
    & \textbf{ESCUELA SUPERIOR DE C\'OMPUTO}
    \end{tabular}
    \end{center}


    \vspace{1.5cm}
    \begin{center}
    \large Trabajo Terminal: \linebreak

    \large \textbf{``API para el Desarrollo de Sistemas de Recomendaci\'on''} \linebreak
    \large 2015-A056

    \end{center}

    \vspace{1.5cm}

    \begin{center}
    Presentan: \linebreak
    \textbf{L\'opez Garduño Blanca Azucena} \linebreak
    \textbf{Castro Esparza Jos\'e Antonio} \linebreak
    \textbf{Bautista de Jesús Héctor Gerardo} \linebreak
    \end{center}

    \vspace{1.5cm}


    %Abstract En este reporte se presenta la documentación técnica del trabajo terminal 2015-A056 titulado “API para el desarrollo de Sistemas de Recomendación” cuyo objetivo es crear una API, la cual permitirá el desarrollo de sistemas de recomendación, dicho sistema contendrá funciones de abstracción, clasificación y análisis de datos , los cuales serán proporcionados por el usuario. Todo esto se desarrollara gracias al cambio de la información recopilada, además de las características de los diferentes objetos y las relaciones importantes entre los objetos similares , además el manejo de los datos ya obtenidos serán mostrados al final para su interpretación y realizar una evaluación. Como caso de estudio, se implementará un sistema de recomendación para platillos y restaurantes de una zona geográfica en específico, dicho sistema trabajará con los gustos de los usuarios finales, validando así la API antes mencionada.  \linebreak

    \textbf{Palabras Clave}:  Inteligencia Artificial, Sistemas de Recomendación, Ingeniería de Software, Machine Learning

    \vspace{1.5cm}
     
    \begin{center}


    Directores: \linebreak
    \textbf{ M. en C. Roc\'io Res\'endiz Muñoz, M. en E. Carlos Silva S\'anchez}

    \end{center}
\end{titlepage}
\pagenumbering{arabic}

%%carta_responsiva
\newpage
%%  COVER
\pagenumbering{Alph}
\begin{titlepage}
    \begin{center}
    \begin{tabular}{r c l}

    \includegraphics[scale=.20]{images/ipn} & \small \textbf{ESCUELA SUPERIOR DE C\'OMPUTO}& \includegraphics[scale=.20]{images/escom}\\ 
    &\small \textbf{SUBDIRECCI\'ON ACAD\'EMICA} \\\\
    &\small \textbf{DEPARTAMENTO DE FORMACI\'ON INTEGRAL E}\\
    &\small \textbf{INSTITUCIONAL}\\\\
    &\small \textbf{COMISI\'ON ACAD\'EMICA DE TRABAJO TERMINAL }\\\\\\
    \end{tabular}
    \end{center}


    \vspace{0.5cm}
   
    \large \raggedleft México, D.F. a 31 de Mayo de 2016.. \linebreak
    
    \vspace{0.5cm}
    \raggedright \small \textbf{DR. FLAVIO ARTURO SÁNCHEZ GARFIAS } \linebreak
    \raggedright \small \textbf{PRESIDENTE DE LA COMISIÓN ACADÉMICA } \linebreak
    \raggedright \small \textbf{DE TRABAJO TERMINAL  } \linebreak
    \raggedright \small \textbf{PRESENTE } \linebreak
    
    \vspace{0.5cm}


    %Abstract En este reporte se presenta la documentación técnica del trabajo terminal 2015-A056 titulado “API para el desarrollo de Sistemas de Recomendación” cuyo objetivo es crear una API, la cual permitirá el desarrollo de sistemas de recomendación, dicho sistema contendrá funciones de abstracción, clasificación y análisis de datos , los cuales serán proporcionados por el usuario. Todo esto se desarrollara gracias al cambio de la información recopilada, además de las características de los diferentes objetos y las relaciones importantes entre los objetos similares , además el manejo de los datos ya obtenidos serán mostrados al final para su interpretación y realizar una evaluación. Como caso de estudio, se implementará un sistema de recomendación para platillos y restaurantes de una zona geográfica en específico, dicho sistema trabajará con los gustos de los usuarios finales, validando así la API antes mencionada.  \linebreak

    Por medio del presente, se informa que los alumnos que integran el \textbf{TRABAJO TERMINAL 2015-A056,} titulado  \textbf{``API para el Desarrollo de Sistemas de Recomendaci\'on''} concluyeron satisfactoriamente su trabajo.\linebreak

    
    Los discos (DVDs) fueron revisados ampliamente por sus servidores y corregidos, 
    cubriendo el alcance y el objetivo planteados en el protocolo original y de acuerdo a los requisitos establecidos por la Comisión que Usted preside. 

    \vspace{0.8cm}
     
    


    \raggedright \textbf{ATENTAMENTE}\\
    \vspace{2.0cm}
    \raggedright \rule{60mm}{0.1mm}\\
    \raggedright \small \textbf {M. en C. Rocío Reséndiz Muñoz}\\
    \raggedright \small \textbf {DIRECTORA DEL TRABAJO} \\
    \raggedright \small \textbf{TERMINAL} \\
    \vspace{-1.6cm}
    \raggedleft \rule {60mm}{0.1mm}\\
    \raggedleft \small \textbf {M. en E. Carlos Silva Sánchez}\\
    \raggedleft \small \textbf {DIRECTOR DEL TRABAJO}  \\
    \raggedleft \small \textbf {TERMINAL} 
   

    \end{titlepage}
\pagenumbering{arabic}