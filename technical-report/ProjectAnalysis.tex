\chapter{Análisis general del proyecto}
Para el desarrollo del sistema en su totalidad, se realizó el siguiente análisis general que define en su totalidad al proyecto.

\section{Características}
Para que el sistema se considere que ha cumplido con los objetivos planteados debe contar con las siguientes características dentro de su funcionalidad.
\begin{itemize}			
	\item El sistema permitirá la búsqueda de platillos.
	\item El sistema permitirá obtener recomendaciones de acuerdo a las características de un platillo.
	\item El sistema obtendrá información de la interacción del usuario a través de evaluaciones explícitas, y conteo de clics implícitos hacia los platillos para obtener recomendaciones personalizadas para dicho usuario.
	\item El sistema permitirá el registro de usuarios finales.
	\item El sistema generará recomendaciones de platillos de acuerdo a la información proporcionada de los usuarios registrados, con base en sus evaluaciones y características.
	\item El sistema permitirá al usuario agregar nuevos platillos que haya consumido en cierto restaurante.
	\item El sistema permitirá al usuario administrar los platillos que agregó. 
\end{itemize}

\section{Restricciones}
Debido a diferentes aspectos, el sistema contará con las siguientes restricciones.
\begin{itemize}
	\item El sistema se verá limitado a las tecnologías empleadas para su desarrollo.
	\item El sistema puede verse limitado debido a la dependencia de fuentes de información externas.
	\item El sistema puede verse limitado en desempeño y precisión debido a la cantidad de información almacenada y a la complejidad del problema a resolver.
\end{itemize} 

\section{Estudio de factibilidad}
Después de definir la problemática presente y presentar la propuesta de solución, es  pertinente realizar un estudio de factibilidad para determinar la infraestructura tecnológica y la capacidad técnica que implica la implementación de la API en cuestión. Este análisis permite determinar las posibilidades de diseñar a API propuesta y su puesta en marcha. 
\\\\
A continuación se describen los aspectos que se toman en cuenta para este análisis.
\subsection{Factibilidad técnica }
Consiste en realizar una evaluación de la tecnología existente; este estudio está destinado a recolectar información sobre los componentes tecnológicos que posee este equipo de desarrollo y la posibilidad de utilizarlos en el desarrollo e implementación de este trabajo y de ser necesario, los requerimientos tecnológicos que deben ser adquiridos su desarrollo e implementación. 
\\\\
Hardware 
\\\\
Para el desarrollo de la API se cuentan con 3 computadoras personales, las cuales cuentan 
con las siguientes características:
\begin{itemize}
	\item MacBook Pro 13: 4 GB DDR3 RAM, Procesador Intel i5 a 2.5 GHz. S.O
	\item HP 8 GB DDR3 RAM, Procesador A10 2.1 GHz. S.O. Ubuntu 14.04.
	\item Acer 8 GB DDR3 RAM Procesador AMD A6 2.0 GHz. Ubuntu 14.04LTS.
\end{itemize}


% Los módulos desarrollados para la primera parte del Trabajo Terminal serán expuestos de 
% manera local, accediendo a ellos a través de una red local. Para la segunda entrega del 
% Trabajo Terminal se tiene planeado exponerlos en un servidor, para lo cual se contará un 
% servicio de hosting. No se incluirá en este documento la opción elegida para el hosting ni el 
% costo del mismo, ya que no se sabe con exactitud cual se elegirá, a pesar de que se han 
% tomado en cuenta varios, no se ha llegado a una decisión.
\newpage
Software  
\\\\
Actualmente el desarrollo de este tipo de proyectos cuenta con la ventaja de existir múltiples tecnologías sobre las cuales es posible desarrollarlo. Por mencionar algunas, el proyecto está pensado para ser desarrollado sobre tecnologías como las listadas a continuación.
\begin{itemize}
	\item Java
	\item Neo4j
	\item Hibernate
	\item JavaScript
	\item HTML
	\item Boostrap
	\item Spring
	\item Maven
\end{itemize}
Estas tecnologías son necesarias para el desarrollo de esta API, cada una cumple con un objetivo específico pero no son indispensables ya que se encuentran muchos otras disponibles con las cuales es posible realizar el desarrollo del proyecto. Para un mayor análisis de opciones disponibles actualmente, revise el apartado de \emph{Tecnologías}.
\\\\
Costos 
\\\\
Los costos que generará el desarrollo de este proyecto se calcularon de la siguiente manera: 
\begin{itemize}
	\item El manejo roles se repartió en el equipo, es decir, cada miembro del equipo cumplió las tareas de analizar, diseñar y desarrollar.  
	\item Nos basamos en un sueldo promedio al cual aspiran estudiantes de la carrera de 
	      Ing. en Sistemas Computacionales, que es de \$78
\end{itemize}
Después de aclarar lo siguiente, los costos operacionales (mano de obra) se calcularon así:
\begin{itemize}
	\item 3 personas que desarrollan diferentes actividades 
	\item Cada uno gana \$78 la hora desarrollando el proyecto
	\item Se toma en cuenta que se trabajan 5 días de la semana 5 horas cada uno de ellos, así: 
\end{itemize}
(\$78) * (5 hrs) * (250 dias) * (3 personas) = \$292,500
 
El precio estimado de este sistema es de \$292,500 MXN solamente tomando en cuenta la mano de obra, sin contemplar nuevos equipos, reuniones, transportes, comida y horas extras. En las siguientes páginas se procede a describir el análisis técnico del proyecto.
\newpage
