\chapter{Prototipo 4: Desarrollo de un sistema híbrido de recomendación de platillos}
  \section{Análisis}
    \subsection{Objetivo}
      Desarrollar un sistema híbrido para la recomendación de platillos que conlleve la demostración de la funcionalidad proporcionada por el API en un caso de estudio particular.

    \subsection{Características}
    \begin{itemize}
      \item El sistema permitirá realizar recomendaciones de platillos de acuerdo a sus características de los mismos (basadas en contenido).
      \item El sistema permitirá realizar recomendaciones de platillos con base en la interacción de los usuarios finales a través de ratings o evaluaciones cuantitativas (por filtrado colaborativo).
      \item El sistema permitirá el registro de usuarios finales, así como su autenticación para el uso de las funcionalidades del sistema.
      \item El sistema permitirá la visualización de los platillos a los usuarios no registrados.
      \item El sistema permitirá la recomendación para los usuarios no registrados a través del manejo de cookies en el navegdor y su interacción con los platillos.
    \end{itemize}

    \subsection{Restricciones}
    \begin{itemize}
      \item El sistema se verá limitado a las características y cantidad de platillos registrados para realizar las recomendaciones.
      \item El sistema no permitirá el registro de platillos a usuarios no registrados.
    \end{itemize}

    Las características del prototipo descritas anteriormente pueden ser visualizadas en la figura que muestra los casos de uso del sistema.

  \section{Diseño}
  
  \section{Resultados}
   