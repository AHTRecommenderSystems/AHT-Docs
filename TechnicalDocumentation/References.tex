\begin{thebibliography}{1}

  \bibitem{1}
    Roger Loaiza. ¿Qué es la inteligencia artificial?. “De la información a la informática”. Obtenido desde \url{http://bvs.sld.cu/revistas/san/vol2_2_98/san15298.htm} el 1 de mayo de 2015.
  \bibitem{2}
    Enrique Herrera-Viedma, Sistemas de recomendaciones: herramientas para el filtrado de información en Internet, 2004. Obtenido desde \url{http://www.upf.edu/hipertextnet/numero-2/recomendacion.html} el 2 de mayo de 2015. 
  \bibitem{3}
     Marcos Merino. ¿Qué es una API y para qué sirve? Obtenido desde: \url{http://www.ticbeat.com/tecnologias/que-es-una-api-para-que-sirve/} el 4 de mayo de 2015. 
  \bibitem{4}
     GEOJson Format Specification, 2008, Available at: \url{http://geojson.org/geojson-spec.html}
  \bibitem{5}
     Sistema Generador de Recomendaciones para una Tienda En-Línea de Videojuegos, 2010. ESCOM IPN. 
  \bibitem{6}
    Gregory Piatetsky. Prediction.io open source machine learning server, 10 de abril de 2014. Obtenido desde: \url{http://www.kdnuggets.com/2014/04/prediction-io-open-source-machine-learning-server.html} el 6 de mago de 2015. 
  \bibitem{7}
    Easyrec, Obtenido desde \url{http://easyrec.org/} el 6 de mayo de 2015 
  \bibitem{8}
    REST API. Obtenido desde: \url{http://easyrec.sourceforge.net/wiki/index.php?title=REST_API_v0.98} el 6 de mayo de 2015. 
  \bibitem{9}
    Netflix. Obtenido desde: \url{https://help.netflix.com/es/node/412} 6 de mayo de 2015.

\end{thebibliography}
\addcontentsline{toc}{chapter}{Bibliografía}
