\chapter{Introducción}
Debido al constante crecimiento de la información que se encuentra disponible en la Internet, en los últimos años el uso de sistemas de recomendación se ha incrementado considerablemente, debido a sus características es utilizado en un sin fin de aspectos, ayudando al comercio electrónico así como en búsquedas de información al mostrar artículos relacionados que pueden ser de interés al usuario final. Esto se logra a través de la aplicación de la Inteligencia Artificial (IA) la cual por medio de ciencias como las ciencias de la computación y la lógica, estudia la creación de entidades que son capaces de resolver problemas por sí mismas utilizando el paradigma de la inteligencia humana. Existen sistemas que cuentan con herramientas de inteligencia artificial, dichos sistemas pueden ejecutar distintos procesos análogos al comportamiento humano, determinando la devolucion de una respuesta por cada entrada mediante una lógica formal. \cite{1}\\

El uso de la IA se ve implicado en los sistemas de recomendación, los cuales son sistemas inteligentes que proporcionan al usuario una serie de sugerencias determinadas por las características de los objetos categorizados y evaluados en el sistema, así como diferentes reglas pertenecientes a la lógica de los algoritmos de selección. Esto resulta en un conjunto de objetos que deberán adaptarse de acuerdo a los cambios en la información en el transcurso del tiempo. Cabe mencionar que los sistemas de recomendación hacen uso de conocimientos relacionados al aprendizaje máquina la cual es una disciplina que trata de que los sistemas aprendan automáticamente, es decir que debe identificar millones de datos en patrones bastante complejos por medio de un algoritmo el cual se encarga de revisar los datos y así es capaz de predecir comportamientos futuros. \cite{2}\\

Debido al problema que conlleva para los programadores obtener los conocimientos necesarios al implementar un sistema de recomendación al mismo tiempo que se intenta resolver un problema en particular, en este Trabajo Terminal se propone desarrollar un conjunto de funciones, reglas, especificaciones y procedimientos (es decir, un API, por sus siglas en inglés (Application Programming Interface ), que brinde dichas funciones con el fin de reducir la curva de aprendizaje del programador al momento de desarrollar sistemas de recomendación. \cite{3} Así, al tiempo que se crean las herramientas necesarias para obtener recomendaciones de artículos, se creará un sistema de recomendación de platillos que verifique la funcionalidad desarrollada en el Trabajo Terminal. 