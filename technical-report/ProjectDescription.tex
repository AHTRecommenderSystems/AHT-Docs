\chapter {Definición del proyecto}
  \section{Planteamiento del problema}
    Actualmente, la cantidad de información que existe en Internet es inmensa, esto nos muestra el creciente problema para los usuarios al encontrar algo que les sea relevante; entre mayor es la cantidad de información, más díficil se vuelve la búsqueda. Es aquí donde los sistemas de recomendación tienen su razón de ser. Tan solo ejemplos claros como Google o Facebook hacen uso de este tipo de sistemas para mostrar los resultados de las búsquedas, o del contenido que se muestra en el \it{feed} de noticias. El uso de los sistemas de recomendación ha permitido el auge de sistemas como los son Amazon, Spotify y Netflix. En general, todo el sistema de comercio electrónico se ha visto beneficiado con el uso de los sistemas de recomendación.\\

    Es por esto, que cada vez más sistemas utilizan este tipo de herramientas para brindar a sus usuarios servicios añadidos que los destaquen entre otras plataformas. Sin embargo, la construcción de un sistema de recomendación no es una tarea sencilla. Requiere de un proceso consciente de análisis para la manera en que los artículos y usuarios se verán relacionados. Así, supone tener un conocimiento de las características propias de un sistema de recomendación, los algoritmos que resuelvan la obtención de recomendaciones y al mismo tiempo, resolver el problema en particular para los artículos y usuarios, lo cual recae en una marcada curva de aprendizaje así como en un mayor tiempo de desarrollo para este tipo de sistemas.\\

  \section{Propuesta de solución}
    Como respuesta a la necesidad creciente del uso de sistemas de recomendación en diferentes ámbitos y plataformas, se propone la creación de una API, definida por un conjunto de funciones, reglas, especificaciones y procedimientos que coadyuven a la creación de sistemas de recomendación para diversos tipos de artículos, con un modelo propuesto por el equipo de trabajo, y la implementación de algoritmos de recomendación para diversos propósitos, como lo son, el filtrado por contenido y el filtrado colaborativo. Así como resolver una problemática para la recomendación de platillos en un área específica de la Ciudad de México que resultaría en la validación de la funcionalidad de la API en cuestión. 

  \section {Objetivo general}
    Diseñar e implementar una API que brinde funciones de abstracción, operación de datos y análisis de los mismos a través de las características de  la información para obtener, por medio de evaluaciones y predicciones, un conjunto de datos que represente una recomendación.

  \section{Justificación}
    El uso de sistemas de recomendación se ha extendido popularmente en los últimos años, teniendo ejemplos en casi cualquier parte de la web, siendo más común su uso en sistemas de comercio electrónico. Sin embargo, el desarrollo de aplicaciones o sistemas que contengan dentro de sus características generar una recomendación requiere un conocimiento especializado en este rubro por parte de los desarrolladores para construir el sistema desde cero, o bien conlleva el uso de componentes comerciales que terminan formando parte del sistema completo. Esto implica que al momento de desarrollar un sistema que resuelva una necesidad a través de un sistema de recomendaciones, el programador debe adquirir los conocimientos necesarios para desarrollar el sistema de recomendaciones al mismo tiempo que intenta resolver el problema de dominio al que se está enfrentando. La API a desarrollar permitirá al programador obtener las herramientas necesarias para la implementación de un sistema de recomendación utilizando conocimientos de la estadística para implementar sistemas de recomendación basados en contenido, colaborativos o híbridos a través de un modelo de datos adecuado que permita realizar dichos procedimientos y así reducir la curva de aprendizaje obligatoria durante la incursión en un dominio de aplicación no conocido, permitiendo al desarrollador concentrar sus esfuerzos en resolver el problema particular de su caso de estudio. \\

    La API brindará funcionalidades propias de los sistemas de recomendación mediante llamadas a bibliotecas que ofrecerán el acceso a las funcionalidades dichas. El desarrollo de esta API que permita proveer la infraestructura mínima necesaria para crear un sistema de recomendación,funge como proyecto integrador de los conocimientos adquiridos durante el estudio de la carrera de Ingeniería en Sistemas Computacionales, para el desarrollo del citado proyecto se requiere hacer uso de los saberes adquiridos por los participantes durante su trayectoria escolar tales como los conocimientos en materia de inteligencia artificial, ingeniería de software, matemáticas discretas, reconocimiento de patrones, programación orientada a objetos, entre otras. Al final el uso conjunto de los conocimientos mencionados terminará en dicha API cuyos beneficios pueden ser vistos de manera inmediata al término de su desarrollo con su verificación y validación en un caso de estudio particular.
