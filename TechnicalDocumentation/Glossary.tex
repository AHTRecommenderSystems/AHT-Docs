\newpage
\section*{Glosario}
\begin{itemize}
  \item \textbf{API}: Application Programming Interface, conjunto de rutinas, protocolos o herramientas para construcción de software.
  \item \textbf{ACID:} Es un acrónimo de Atomicity, Consistency, Isolation and Durability: Atomicidad, Consistencia, Aislamiento y Durabilidad en español.
  \item \textbf{Atomicidad:}Si una operación consiste en una serie de pasos, todos ellos ocurren o ninguno, es decir, las transacciones son completas.
  
  \item \textbf{Atributo}: Los atributos son las características individuales que diferencian un objeto de otro y determinan su apariencia, estado u otras cualidades.Los atributos se guardan en variables denominadas de instancia, y cada objeto particular puede tener valores distintos para estas variables.
  \item \textbf{Aislamiento:} es la propiedad que asegura que una operación no puede afectar a otras. Esto asegura que la realización de dos transacciones sobre la misma información sean independientes y no generen ningún tipo de error.  Esta propiedad define cómo y cuándo los cambios producidos por una operación se hacen visibles para las demás operaciones concurrentes.
  \item \textbf{Base De Datos:} Es un conjunto de datos pertenecientes a un mismo contexto y almacenados sistemáticamente para su posterior uso.
  \item  \textbf{Caso De Uso:} Es una descripción de los pasos o las actividades que deberán realizarse para llevar a cabo algún proceso. 
  \item \textbf{Consistencia:} Es la propiedad que asegura que sólo se empieza aquello que se puede acabar. Por lo tanto se ejecutan aquellas operaciones que no van a romper las reglas y directrices de Integridad de la base de datos. La propiedad de consistencia sostiene que cualquier transacción llevará a la base de datos desde un estado válido a otro también válido. 
  \item \textbf{Durabilidad:} Persistencia. Es la propiedad que asegura que una vez realizada la operación, ésta persistirá y no se podrá deshacer aunque falle el sistema y que de esta forma los datos sobrevivan de alguna manera.
  \item \textbf{Entidad:} Representa una cosa u objeto del mundo real con existencia independiente, es decir, se diferencia únicamente de otro objeto o cosa, incluso siendo del mismo tipo, o una misma entidad.
  \item \textbf{Framework: }Es una estructura conceptual y tecnológica de soporte definido, normalmente con artefactos o módulos concretos de software, que puede servir de base para la organización y desarrollo de software. 
   \item \textbf{Grafo:} Estructura la información en forma de nodos y relaciones.
   \item \textbf{Prototipo:} Es una representación de un sistema, aunque no es un sistema completo, posee las características del sistema final o parte de ellas.
   \item \textbf{Web Scrapping}:  Técnica implemendatada mediante programas de software para extraer información de sitios web.
\end{itemize}
\addcontentsline{toc}{chapter}{Glosario}