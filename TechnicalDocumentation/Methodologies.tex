\clearpage
\chapter{Metodología}
\section{Descripción}
  \paragraph{Para ese proyecto se plantea usar una metodología de prototipado evolutivo, que se caracteriza por que en su modelo de trabajo un prototipo es construido, probado y finalmente reconstruido las veces que sea necesario hasta que un prototipo aceptable es finalmente alcanzado del cual el sistema completo o producto puede ser totalmente desarrollado.}

  \paragraph{Este modelo funciona bien en escenarios donde no se conocen por completo los requerimientos. Así los prototipos, modelos de software con una funcionalidad limitada, permiten al usuario evaluar los propósitos del desarrollador y probarlos antes de su implementación.}
  \paragraph{También ayuda a entender los requerimientos específicos del usuario y que no pudieron haber sido considerados por el desarrollador durante el diseño del sistema.}

%\subsection{Implementación del modelo de prototipos}
%\begin{itemize}
  %\item Los nuevos requerimientos del sistema se definen con el mayor detalle posible.
  %\item Un diseño preeliminar es creado para el nuevo sistema. 
%  \item Un primer prototipo del nuevo sistema es construido para el diseño preeliminar que representa una aproximación a las características del producto final.
%  \item El usuario evalua el primer prototipo, notando sus fortalezas y debilidades, qué necesita agregarse y que debería removerse. El usuario recoge las observaciones de los usuarios. 
%  \item El primer prototipo es modificado, basado en los comentarios hechos por el usuario, y un segundo prototipo del nuevo sistema es construido.
  %\item El segundo prototipo es evaluado del mismo modo que el primero.
%  \item Los pasos anteriores son iterados tantas veces como sea necesario, hasta que los usuarios consideren que el prototipo representa al producto final deseado.
  %\item El sistema final es construido, basado en el prototipo final.
%  \item El sistema final es evaluado y probado a fondo. Existe una rutina de mantenimiento continuo para prevenir errores a gran escala.
%\end{itemize}

\section{Prototipos esperados}
  \paragraph{De acuerdo al modelo de desarrollo elegido, se han establecido los siguientes prototipos y sus alcances:}
  \paragraph{Versión: Alcance}
  \paragraph{Prototipo 1: Obtener y abstraer los datos a utilizar.}
  \paragraph{Prototipo 2: Visualizar los datos en recomendaciones topN.}
  \paragraph{Prototipo 3: Obtener recomendaciones para diversos fines. }
  \paragraph{Prototipo 4: Desarrollar un sistema de recomendación de platillos y restaurantes.}

